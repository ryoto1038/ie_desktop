% !TEX root = MasterPaper.tex
\chapter{先行研究}
\thispagestyle{fancy} % このページのみ
\lhead{}
\chead{}
\rhead{}
\lfoot{} 
\cfoot{\thepage}  
\rfoot{}
%

\section{遺伝的アルゴリズム}
\label{sec2.1}

\subsection{遺伝的アルゴリズムの概要}
\label{sec2.1.1}

GAとは選択淘汰や突然変異など生物進化の仕組みを模範した最適化アルゴリズムである.GAは1975年にJ.H.Hollandnによって提案された手法である.GAの枠組みはとても簡単であり,与えられた最適化問題の評価関数に対して,いくつかの乱数と単純な記号処理を用いるだけで求めることができる.また,GAは他の最適化アルゴリズムより比較的少ない計算量で最適解を求めることができる.

GAのフローチャートを図2.1に示す.GAでは,生成した遺伝子列に対して,選択,交叉,突然変異といった生物進化の仕組みを模した処理を行う.問題の解候補を生物集団の各個体と予備,各個体のパラメータを遺伝子と呼ぶ.

以下に図2.1の具体的な流れについて述べる.

\begin{description}
\item[ (1) ]遺伝子型の決定

GAの対象となる問題を遺伝子列に変換する

\item[ (2) ]初期遺伝子集団の決定

(1)で決められた遺伝子型で,要素の異なるさまざまな個体をランダムに発生させる.

\item[ (3) ]適応度評価

生成された遺伝子集団に対しての評価を行い,各個体の適応度をあらかじめ定められた計算方法と評価で算出する.

\item[ (4) ]選択処理

遺伝子集団中における各個体の適応度に基づいて,交叉処理を行う個体を選択する.

\item[ (5) ]交叉処理

(4)で選択された2つの個体間で遺伝子を組み替えて新しい個体を発生させる.

\item[ (6) ]突然変異処理

遺伝子のある部分を特定の確率で強制的に変化させる.

\item[ (7) ]終了条件(遺伝子集団の評価)

生成された次世代の遺伝子集団が,GA処理を終了するための評価基準を満足しているかどうかを確認する.
\end{description}



\subsection{各個体の評価処理}
\label{sec2.1.2}
  
各個体の評価処理は,あらかじめ定めた適応度により,各個体の適応度を求める操作である.本処理は遺伝子型と設定されている記号列を実際の評価型にデコーディングして,その表現型と設定されている環境との適応を判定することによって行われる.個体間の適応度の差が激しい場合,選択処理時に適応度の高い個体が選ばれる確率が非常に高くなる.そしてその個体の遺伝子が集団内に増加することにより短時間で探索が収束してしまうためより適応度の高い遺伝子の探索が困難となる.そこで,適応度の値を直接反映させるのではなく,関数を用いて変換してから,選択に反映させるスケーリングを行う.スケーリング関数の例を示す.表2.1において$f$は元の適応度,$f'$は新たな適応度である.べき乗スケーリングにおいて,$k$はスケーリング指数と呼ばれる.また,シグマ切断の関数において,$\sigma$は適応度の標準偏差,$\bar{f}$は適応度の平均値である.

\begin{table}[!ht]
\caption{スケーリング関数の例}
\label{tb:sk}
\begin{center}
\begin{tabular}{|c||c|}\hline
スケーリング & 関数 \\ \hline
線形スケーリング & $f'=af+b$ \\ \hline
べき乗スケーリング & $f'=f^{k}$ \\ \hline
シグマ切断 & $f'=f-( \bar{f} - c \times \sigma )$ \\ \hline
\end{tabular}
\end{center}
\end{table}

\subsection{選択処理}
\label{sec2.1.3}

\subsection{交叉処理}
\label{sec2.1.4}

\subsection{突然変異処理}
\label{sec2.1.5}




\newpage

\section{対話型進化計算}
\label{sec2.2}

先行研究において,臨場感を演出するためには,「その場にいるような感覚」と「興奮」の2つの要素が必要であると述べられている.
スポーツ観戦における臨場感は,「その場にいるような感覚」を集団が生み出す一体感と捉え,「興奮」は感情伝播を用いることで,2つの要素を満たすことができると考えられる.

そこで本節では,スポーツ観戦において臨場感の演出に必要な要素だと考えられる,集団が生み出す一体感と感情伝播について述べる.

\subsection{対話型進化計算の概要}
\label{sec2.2.1}

「一体」とは,ある個人と他者の「心的内容・心理状態」が一致,あるいは同じである状態を指す.そして「一体感」とは,「一体」を意識的,あるいは無意識的に感じている心的状態,心理を指す\cite{ittai}.一体感を得ることで,人々はよりその場にいるように感じることができるため,場面への没入感が増し,より大きな臨場感を得ることができると考えられる.スポーツ観戦においては,人々は自身と他者が同じチームを応援し,互いに感情表出を行うことで一体感を得ることができる.
    
\subsection{対話型遺伝的アルゴリズム}
\label{sec2.2.2}

「感情伝播」とは,他人の動作・表現・態度・発言を無意識に真似したり,同調したりすることで同じような気分になることである\cite{denpa}.例えば,周囲からポジティブな感情表現を刺激として受けた人は,ポジティブな感情を抱く傾向を持つ.また,感情伝播は物理的に近い人に起こりやすい傾向があると言われている.そのため,スポーツ観戦時の人が密集した空間では特に発生しやすいと考えられる.先行研究においても,スポーツ観戦を行う観客がプレーに反応して,強い感情伝播を起こすことが分かっている\cite{jyodo}.すなわち,感情伝播により,周囲の興奮から影響を受けることで興奮が高まり,より高い臨場感を得ることができると考えられる.

\newpage

\section{感性とロボット}
\label{sec2.3}

\subsection{ロボットの進化}
\label{sec2.3.1}

人間の代わりに過酷な作業を行う産業用ロボットは,1960年代に開発が進み,1980年代から工場の生産ラインなどに導入され,普及し始めた.この時代のロボットは作業場所も固定されており,単純な繰り返し作業をするというものだった.その後,ロボットの認識や記憶,学習などの能力向上により,様々な機能を持ったロボットが登場した.例えば,実際に利用されている警備ロボットや掃除ロボットは,作業場所が固定されておらず,屋内外を自由に動き回るような機能を有する.

現在では,産業用ロボットだけではなく,非産業用であるパーソナルロボットと呼ばれるロボットが登場している.パーソナルロボットとは,人間と密接に関わり合うことで,人間に快適な生活を積極的に提供してくれるロボットである\cite{perso}.例として,高齢者や障害者の手助けをする介護ロボットや,人間の心を和ませる動物の動きを模したペットロボット,人間とコミュニケーションをとるコミュニケーションロボットなどが存在している.




\subsection{性格特性とハンドジェスチャ}
\label{sec2.3.2}

近年開発されている家庭用ロボットや案内ロボットは,人と直接関わる機会が多いロボットである.これらのロボットが,人間の生活環境に適応するためには,人間同士のコミュニケーションを模倣した振る舞いを行う必要がある.例えば,人間同士のコミュニケーションでは,言語情報のみならず非言語情報のやり取りも重要な意味を持つ\cite{gengo}.非言語情報のうちでも特に,感情や情動などの心理的情報が重要である.また,ロボットは人間の感性表現を理解するだけでなく,感性を表現する機能も必要である.これらの感性に関する処理をロボットに導入することで,コミュニケーションによる親しみを生み出すことができる.このように,ロボットに感性を付け加えることで,ロボットの親しみやすさは増加すると言われている.

例えば,小林らは,ロボットが人の感情を認識し,それに対する反応を人にわかりやすいように表示伝達するためのアクティブ・ヒューマン・インターフェースを提唱し,この表示伝達方法として顔ロボットの顔表情による感性表現について調査を行っている\cite{gengo}.






\vspace{1cm}
\begin{figure}[!h]
 \begin{center}
  \centering
  \label{fig:kansei}
 \end{center}
\end{figure}

