% !TEX root = MasterPaper.tex
\chapter{先行研究}
\thispagestyle{fancy} % このページのみ
\lhead{}
\chead{}
\rhead{}
\lfoot{} 
\cfoot{\thepage}  
\rfoot{}
%

\section{遺伝的アルゴリズム}
\label{sec2.1}

\subsection{遺伝的アルゴリズムの概要}
\label{sec2.1.1}

GAとは選択淘汰や突然変異など生物進化の仕組みを模範した最適化アルゴリズムである.GAは1975年にJ.H.Hollandnによって提案された手法である.GAの枠組みはとても簡単であり,与えられた最適化問題の評価関数に対して,いくつかの乱数と単純な記号処理を用いるだけで求めることができる.また,GAは他の最適化アルゴリズムより比較的少ない計算量で最適解を求めることができる.

GAのフローチャートを図2.1に示す.GAでは,生成した遺伝子列に対して,選択,交叉,突然変異といった生物進化の仕組みを模した処理を行う.問題の解候補を生物集団の各個体と予備,各個体のパラメータを遺伝子と呼ぶ.

以下に図2.1の具体的な流れについて述べる.

\begin{description}
\item[ (1) ]遺伝子型の決定

GAの対象となる問題を遺伝子列に変換する

\item[ (2) ]初期遺伝子集団の決定

(1)で決められた遺伝子型で,要素の異なるさまざまな個体をランダムに発生させる.

\item[ (3) ]適応度評価

生成された遺伝子集団に対しての評価を行い,各個体の適応度をあらかじめ定められた計算方法と評価で算出する.

\item[ (4) ]選択処理

遺伝子集団中における各個体の適応度に基づいて,交叉処理を行う個体を選択する.

\item[ (5) ]交叉処理

(4)で選択された2つの個体間で遺伝子を組み替えて新しい個体を発生させる.

\item[ (6) ]突然変異処理

遺伝子のある部分を特定の確率で強制的に変化させる.

\item[ (7) ]終了条件(遺伝子集団の評価)

生成された次世代の遺伝子集団が,GA処理を終了するための評価基準を満足しているかどうかを確認する.
\end{description}



\subsection{各個体の評価処理}
\label{sec2.1.2}
  
各個体の評価処理は,あらかじめ定めた適応度により,各個体の適応度を求める操作である.本処理は遺伝子型と設定されている記号列を実際の評価型にデコーディングして,その表現型と設定されている環境との適応を判定することによって行われる.個体間の適応度の差が激しい場合,選択処理時に適応度の高い個体が選ばれる確率が非常に高くなる.そしてその個体の遺伝子が集団内に増加することにより短時間で探索が収束してしまうためより適応度の高い遺伝子の探索が困難となる.そこで,適応度の値を直接反映させるのではなく,関数を用いて変換してから,選択に反映させるスケーリングを行う.スケーリング関数の例を示す.表2.1において$f$は元の適応度,$f'$は新たな適応度である.べき乗スケーリングにおいて,$k$はスケーリング指数と呼ばれる.また,シグマ切断の関数において,$\sigma$は適応度の標準偏差,$\bar{f}$は適応度の平均値である.

\begin{table}[!ht]
\caption{スケーリング関数の例}
\label{tb:sk}
\begin{center}
\begin{tabular}{|c||c|}\hline
スケーリング & 関数 \\ \hline
線形スケーリング & $f'=af+b$ \\ \hline
べき乗スケーリング & $f'=f^{k}$ \\ \hline
シグマ切断 & $f'=f-( \bar{f} - c \times \sigma )$ \\ \hline
\end{tabular}
\end{center}
\end{table}

\subsection{選択処理}
\label{sec2.1.3}

各個体の選択処理は,集団内での適応度の分布にしたがって,交叉を行う個体の生存分布を決定する操作である.以下に,基本的な選択処理方法である適応度比例方式とエリート保存方式について述べる.

\begin{description}
\item[ (1) ]適応度比例方式

適応度比例方式はルーレット選択方式とも呼ばれ,各個体の子孫がその適応度に比例した確率で選択される方法である.適応度比例方式の最も簡単な実現方法は,適応度に基づいた円グラフをルーレットとして回し,ルーレットの玉が入った領域の個体を選び出すというものである.式(\ref{eq:2.1})に重み付けルーレット方式の式を示す.



\begin{equation}
\vspace{1.0cm}
{p_i}=\frac{f_i}{\displaystyle\sum_{i=1}^{n}f_i}
\label{eq:2.1}
\vspace{-0.5cm}
\end{equation}
式(\ref{eq:2.1})において,$p_i$は$i$番目の個体が親として選ばれる確率,$f_i$は$i$番目の個体適応度,$n$は個体数である.

\item[ (2) ]エリート保存方式

エリート保存方式は,個体の中で最も適応度の高い個体はそのまま次世代に残すという方法である.最も適応度の高い個体は交叉や突然変異による変形を受けないという利点がある.ただし場合によってはあまり良くない遺伝子が急速に集団中に広がる.つまり局所的最適解に収束する危険性がある.そのため一般的には他の選択方法と組み合わせて用いられる.

\end{description}

\subsection{交叉処理}
\label{sec2.1.4}

交叉処理は,2つの染色体間で,遺伝子を組み替えて,新たな個体を発生させる操作である.交叉処理で用いられる基本的な手法である一点交叉,複数点交叉,一様交叉について遺伝子列がビット列である場合の例を図に示す.
%
\newpage

\begin{description}
\item[ (1) ]一点交叉

図に示すように,遺伝子上に交叉位置をランダムで一点決定する.交叉位置で,親1と親2の遺伝子を入れ替え,子1と子2を発生させる.

\item[ (2) ]複数点交叉

図に示すように,遺伝子上に複数の交叉位置を作る.交叉位置ごとに親1と親2の遺伝子を入れ替え,子1と子2を発生させる.

\item[ (3) ]一様交叉

図に示すように,マスクパターンをランダムに生成する.そして,2つの親個体に対し,そのマスクのビットが0であるなら親1を,1であるなら親2の遺伝子をコピーして子1を発生させる.同時に逆のコピーを行い,子2を発生させる.

\end{description}

\newpage


\subsection{突然変異処理}
\label{sec2.1.5}

突然変異処理は,遺伝子のある部分を特定の確率で強制的に変化させる操作である.この操作により遺伝子集団に多様性をもたせることでより良い解を持つ個体の発生を促す.ただし,突然変異の確率を大きくしすぎると解の収束が遅くなる.

\begin{description}

\item[ (1) ]転座方式

図に転座方式を示す.転座方式は,遺伝子の一部が同じ遺伝子の他の部分,または他の遺伝子上に位置を移す処理である.

\item[ (2) ]逆位方式


図に逆位方式を示す.逆位方式は,部分的に遺伝子の配列順序を入れ替える処理である.


\end{description}


\newpage

\section{対話型進化計算}
\label{sec2.2}

\subsection{対話型進化計算の概要}
\label{sec2.2.1}

感性情報を扱うまでは情報処理におけるシステムの最適化は望ましい出力パラメータを調節することであった.ほとんどの場合,システムの望ましい出力は定量的なものであり,主に誤差最小化規範などを代表とする各種最適化手法が開発されてきた.

しかし近年では情報処理におけるシステムの最適化の対象は感性情報にまで及んでいる.その場合,望ましい出力がユーザが好む音楽や画像などユーザ自身が主観的に決めるものであるため望ましい出力を数値的に導くことは困難である.このようなシステムの最適化において,従来と異なる最適化手法が必要となる.その方法の一つとして,ユーザの評価系の代替モデルを作成し,これを従来の最適化システムに組み込むことで数値的に導き出せるようにすることが考えられる.しかし,個人の主観や好みに依存する情報に対応できる正確な代替モデルを構築することは非常に難しい.そこで,ユーザ自身が最適化系に組み込み,ユーザ本人の評価に基づいてコンピュータに最適化させる方法が考えられる.このようにユーザとコンピュータがコミュニケーションをとり,ユーザの主観的評価に基づいて最適化を行う手法のうち,ECを用いる手法をIECと呼ぶ.IECは人間の主観的評価に基づいて最適化を行うため,感性を具現化する技術といえる.

代表的なEC 技術として,GA,タブーサーチ法(Taboo Serch: TS),遺伝的プログラミング (Genetic Programming: GP),進化戦略 (Evolution Strategy: ES),進化的プログラミング (Evolutionary Programming: EP),粒子群最適化 (Particle Swarm Optimization: PSO) などがある.
    
\subsection{対話型遺伝的アルゴリズム}
\label{sec2.2.2}

本研究では,GAにおける評価系に人の感性を組み込んだ手法であるIGAを用いる.
IGAは,GA処理における解候補評価をユーザが自らの感性に基づいて行い,解候補を最適化する手法である.

IGAの基本的な処理の流れを図に示す.まず初期解候補を生成する.次に,解候補をユーザに提示し,ユーザが主観的評価を行う.ユーザによる評価が終了すると,ユーザによって与えられた各解候補の評価をもとに選択・交叉・突然変異等の処理が行われ,新たな解候補を生成し,再びユーザに提示する.このような処理を繰り返し,ユーザの感性に合う解候補を生成する.

IGAは,ユーザ評価を取り入れた進化計算手法であるIECにおいてGAの進化計算アルゴリズムを適応させた手法である.すなわち,IGAとは通常のGAにおいて適応度を決定する処理をユーザが行う手法である.ユーザの主観的評価が適応度に反映されるため,IGAは特にユーザの感覚的,直観的な評価が必要とされる問題の解決に多く用いられる.また一般的にIGAはユーザの評価回数が多いほど最適化性能が高くなる傾向にある.したがって,IGAを用いた最適化システムが十分な最適化性能を発揮するためには多くのユーザ評価が必要となる.しかし,ユーザに多くの評価を求めることは,ユーザの評価負担の増加に繋がる可能性がある.



\newpage

\section{感性とロボット}
\label{sec2.3}

\subsection{ロボットの進化}
\label{sec2.3.1}

人間の代わりに過酷な作業を行う産業用ロボットは,1960年代に開発が進み,1980年代から工場の生産ラインなどに導入され,普及し始めた.この時代のロボットは作業場所も固定されており,単純な繰り返し作業をするというものだった.その後,ロボットの認識や記憶,学習などの能力向上により,様々な機能を持ったロボットが登場した.例えば,実際に利用されている警備ロボットや掃除ロボットは,作業場所が固定されておらず,屋内外を自由に動き回るような機能を有する.

現在では,産業用ロボットだけではなく,非産業用であるパーソナルロボットと呼ばれるロボットが登場している.パーソナルロボットとは,人間と密接に関わり合うことで,人間に快適な生活を積極的に提供してくれるロボットである\cite{perso}.例として,高齢者や障害者の手助けをする介護ロボットや,人間の心を和ませる動物の動きを模したペットロボット,人間とコミュニケーションをとるコミュニケーションロボットなどが存在している.




\subsection{性格特性とハンドジェスチャ}
\label{sec2.3.2}

近年開発されている家庭用ロボットや案内ロボットは,人と直接関わる機会が多いロボットである.これらのロボットが,人間の生活環境に適応するためには,人間同士のコミュニケーションを模倣した振る舞いを行う必要がある.例えば,人間同士のコミュニケーションでは,言語情報のみならず非言語情報のやり取りも重要な意味を持つ\cite{gengo}.非言語情報のうちでも特に,感情や情動などの心理的情報が重要である.また,ロボットは人間の感性表現を理解するだけでなく,感性を表現する機能も必要である.これらの感性に関する処理をロボットに導入することで,コミュニケーションによる親しみを生み出すことができる.このように,ロボットに感性を付け加えることで,ロボットの親しみやすさは増加すると言われている.

例えば,小林らは,ロボットが人の感情を認識し,それに対する反応を人にわかりやすいように表示伝達するためのアクティブ・ヒューマン・インターフェースを提唱し,この表示伝達方法として顔ロボットの顔表情による感性表現について調査を行っている\cite{gengo}.






\vspace{1cm}
\begin{figure}[!h]
 \begin{center}
  \centering
  \label{fig:kansei}
 \end{center}
\end{figure}

