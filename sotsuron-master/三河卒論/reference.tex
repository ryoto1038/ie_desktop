% !TEX root = MasterPaper.tex
\renewcommand{\bibname}{参考文献}
\addcontentsline{toc}{chapter}{参考文献}

\begin{thebibliography}{27} %{}内に参考文献の総数を書く

%序論



%2章
\bibitem{GA}
北野宏明,遺伝的アルゴリズム,産業図書,1993


\bibitem{GA2}
北野宏明,遺伝的アルゴリズム[4],産業図書,2000.


\bibitem{IEC}
Hideyuki Takagi,“Interactive evolutionary computation: fusion of the capabilities of EC optimization and human evaluation”,Proceedings of the IEEE,Vo.89,Issue.9,pp.1275-1296,2001.


\bibitem{ロボットの性格特性1}
Elisabeth Andr'e, Thomas Rist, Susanne van Mulken, Martin Klesen, and Stephan Baldes. The Automated Design of Believable Dialogues for Animated Presentation Teams, page 220–255. MIT Press, Cambridge, MA, USA, 2001.

\bibitem{ロボットの性格特性2}
Ning Wang, W. Johnson, Richard Mayer, Paola Rizzo, Erin Shaw, and Heather Collins. The politeness effect: Pedagogical agents and learning gains. pages 686–693, 01 2005.


\bibitem{ロボットの性格特性3}
Scott McQuiggan, Bradford Mott, and James Lester. Modeling self-efficacy in intelligent tutoring systems: An inductive approach. User Model. User-Adapt. Interact., 18:81–123, 02 2008.

\bibitem{ハンドジェスチャ強調}
Justine Cassell, Tim Bickmore, Lee Campbell, Hannes Vilhjalmsson, and Hao Yan. Human conversation as a system framework: Designing embodied conver- sational agents. In Justine Cassell, Joseph Sullivan, Scott Prevost and Elizabeth Churchill, editors, Embodie Conversational Agents, pages 29–63. MIT Press, 2000

\bibitem{性格特性推測}
Oya Aran and Daniel Gatica-Perez. One of a kind: Inferring personality impressions in meetings. In ICMI, page 11–18, 2013


\bibitem{ジェスチャ生成}
中野有紀子, et al. "性格特性を表現するエージェントジェスチャの生成." ヒューマンインタフェース学会論文誌 23.2 (2021): 153-164.


\bibitem{ハンドジェスチャ指標1}
Mark L Knapp and Gerald R Miller. Communicator characteristics and behavior. Handbook of Interpersonal Communication, pages 103–161, 1994.


\bibitem{ハンドジェスチャ指標2}
Harrison Jesse Smith and Michael Neff. Understanding the impact of animated gesture performance on personality perceptions. ACM Trans. Graph., 36(4), July 2017.

\bibitem{ハンドジェスチャ指標3}
Kevin Frank. Posture & Perception in the Context of the Tonic Function Model of Structural Integration: An Introduction. IASI Yearbook, 2007.

\bibitem{ハンドジェスチャ外向性}
Kim, Heeyoung et al. “Personality design of sociable robots by control of gesture design factors.” RO-MAN 2008 - The 17th IEEE International Symposium on Robot and Human Interactive Communication (2008): 494-499.


%3章



%4章














\end{thebibliography}