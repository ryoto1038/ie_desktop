% !TEX root = MasterPaper.tex
\renewcommand{\bibname}{参考文献}
\addcontentsline{toc}{chapter}{参考文献}

\begin{thebibliography}{27} %{}内に参考文献の総数を書く

%序論

\bibitem{hamasuta}
% バーチャルハマスタ,``https:\slash\slash{}www.au.com\slash{}sports\slash{}baseball\slash{}articles\_virtual\_hamasta202103\slash{}'',
バーチャルハマスタ,``https://www.au.com/sports/baseball/articles\_virtual\_\\hamasta202103/'',
最終閲覧日:2022/2/1.

\bibitem{healsio}
ウォーターオーブン ヘルシオ:シャープ,``http://healsio.jp/'',最終閲覧日:2022/2/1.

\bibitem{go}
D.Silver,A.Huang,C.J.Maddison,A.Guez,L.Sifre,G.v.d.Driessche,J.Schrittwieser,I.Antonoglou,V.Panneershelvam,M.Lanctot,S.Dieleman,
D.Grewe,J.Nham,N.Kalchbrenner,I.Sutskever,T.Lillicrap,M.Leach,K.Kavukcuoglu,T.Graepel,D.Hassabis,
``Mastering the game of Go with deep neural networks and tree search'',Nature 529, pp.484-489, 2016.

\bibitem{toyota}
トヨタ-パーソナルロボット,``https://www.toyota.co.jp/jpn/tech/partner\_robot/'',最終閲覧日:2022/2/1.

\bibitem{aibo}
aibo,``http://aibo.sony.jp/'',最終閲覧日:2022/2/1.

\bibitem{deep}
松尾豊,``人工知能は人間を超えるか --ディープランニングの先にあるもの--'',KADOKAWA/中経出版,2015.

\bibitem{kao}
小林宏,原文雄,内田豪,大野宗久,``アクティブ・ヒューマン・インターフェース(AHI)のための顔ロボットの研究 
-顔ロボットの機構と6基本表情の表出-'',日本ロボット学会誌,vol.12,no.1,pp.155-163,1994.

\bibitem{syuwa}
三輪敬之,``手話ロボットハンド'',日本ロボット学会誌,vol.7,no.3,pp.113,1989.

\bibitem{higengo}
菅野重樹,渋谷恒司,``非言語コミュニケーションのための人間形ロボット'',日本ロボット学会誌,vol.15,no.7,pp.975-978,1997.

%2章

\bibitem{rinjyo1}
谷口高士,``臨場感の構成概念と評価について'',大阪学院大学,人文自然論叢,
第69-70号,2015.

\bibitem{lombard}
M.Lombard,T.Ditton,``At the heart of it all: The concept of presence'',Journal of Computer-Mediated Communication,3,online,1997.

\bibitem{ando}
安藤広志,``超臨場感に対する五感・認知'',原島博(監修)・映像情報メディア学会(編),超臨場感システム,オーム社,194-199,2010.

\bibitem{rinjyo2}
寺本渉,吉田和博,浅井暢子,日高聡太,行場次朗,鈴木陽一,``臨場感の素朴な理解'',
日本バーチャルリアリティ学会論文誌,Vol.105,No.1,2010.

\bibitem{rinjyo3}
橋本泰裕,中田大貴,``歓声量から観客を興奮させるプレーを評価する'',体力測定評価研究,2019.

\bibitem{ittai}
塹江清志,水野和夫,塹江光子,``「一体感」と「断絶感」'',名古屋工業大学紀要,50巻,p.185-190,1999-03-31.

\bibitem{denpa}
安藤和代,``ポジティブなクチコミにおいて非言語要素が誘発する感情伝播効果'',千葉商大論叢,51巻1号,pp.63-82,2013.

\bibitem{jyodo}
F.M.Gotz,S.Stiegert,T.Ebert,P.J.Rentfrow,D.Lewetz,`` What Drives Our Emotions When We Watch Sporting Events? An ESM Study on the Affective Experience of German Spectators During the 2018 FIFA World Cup''
,Collabra,Psychology,6(1),15,DOI,2020.

\bibitem{perso}
伊藤俊樹,長田純一,山口智治,藤田善弘,市川玲,江角彩,近藤安津美,近藤龍彰,
``パーソナルロボット「パペロ」に対する好意度尺度の作成'',日本デザイン学会研究発表大会概要集,日本デザイン学会,
第58回研究発表大会,一般社団法人,日本デザイン学会,2011.

\bibitem{gengo}
後藤みの理, 加納政芳, 加藤昇平, 國立勉, 伊藤英則, ``感性ロボットのための感情領域を用いた表情生成'', 
人工知能学会論文誌, 21巻1号G, 2006.

\bibitem{tekiou}
高岡勇紀, 尾関基行, 岡夏樹, ``人とロボットの相互適応過程を考慮したロボットの学習方法'', 
電子情報通信学会技術研究報告ヒューマンコミュニケーション基礎研究会, Vol114, 
No517, HCS2014-125, pp.75-80, 2015.

\bibitem{rabot}
らぼっと,``https://lovot.life/'',最終閲覧日:2022/1/7.

%3章

\bibitem{kyusute}
キューステ!, ``https://sports-station.jp/'', 最終閲覧日:2022/1/23.

\bibitem{unity}
Unity, ``https://unity.com/'', 最終閲覧日:2022/1/23.

\bibitem{bar}
Sports Bar 01,``https://assetstore.unity.com/packages/3d/environments/sport-bar-01-194104'',最終閲覧日:2022/1/23.

\bibitem{lilrobot}
Lil Robots with changeable expressions,``https://assetstore.unity.com/packages/\\3d/characters/robots/lil-robots-with-changeable-expressions-147408'',\\最終閲覧日:2022/1/23.

\bibitem{vive}
VIVE, ``https://www.vive.com/'', 最終閲覧日:2022/1/23.

%4章

\bibitem{patv}
パ・リーグTV,``https://tv.pacificleague.jp/ptv/pc/'',最終閲覧日:2022/1/29.




\end{thebibliography}